%%%%%%%%%%%%%%%%%%%%%%%%%%%%%%%%%%%%%%%%%
% Frequently Asked Questions
% LaTeX Template
% Version 1.0 (22/7/13)
%
% This template has been downloaded from:
% http://www.LaTeXTemplates.com
%
% Original author:
% Adam Glesser (adamglesser@gmail.com)
%
% License:
% CC BY-NC-SA 3.0 (http://creativecommons.org/licenses/by-nc-sa/3.0/)
%
%%%%%%%%%%%%%%%%%%%%%%%%%%%%%%%%%%%%%%%%%

\documentclass[11pt]{article}

\usepackage[margin=1in]{geometry} % Required to make the margins smaller to fit more content on each page
\usepackage[linkcolor=blue]{hyperref} % Required to create hyperlinks to questions from elsewhere in the document
\hypersetup{pdfborder={0 0 0}, colorlinks=true, urlcolor=blue} % Specify a color for hyperlinks
\usepackage{todonotes} % Required for the boxes that questions appear in
\usepackage{tocloft} % Required to give customize the table of contents to display questions
\usepackage{microtype} % Slightly tweak font spacing for aesthetics
\usepackage{palatino} % Use the Palatino font

\setlength\parindent{0pt} % Removes all indentation from paragraphs

% Create and define the list of questions
\newlistof{questions}{faq}{\large List of Frequently Asked Questions} % This creates a new table of contents-like environment that will output a file with extension .faq
\setlength\cftbeforefaqtitleskip{4em} % Adjusts the vertical space between the title and subtitle
\setlength\cftafterfaqtitleskip{1em} % Adjusts the vertical space between the subtitle and the first question
\setlength\cftparskip{.3em} % Adjusts the vertical space between questions in the list of questions

% Create the command used for questions
\newcommand{\question}[1] % This is what you will use to create a new question
{
\refstepcounter{questions} % Increases the questions counter, this can be referenced anywhere with \thequestions
\par\noindent % Creates a new unindented paragraph
\phantomsection % Needed for hyperref compatibility with the \addcontensline command
\addcontentsline{faq}{questions}{#1} % Adds the question to the list of questions
\todo[inline, color=green!40]{\textbf{#1}} % Uses the todonotes package to create a fancy box to put the question
\vspace{1em} % White space after the question before the start of the answer
}

% Uncomment the line below to get rid of the trailing dots in the table of contents
%\renewcommand{\cftdot}{}

% Uncomment the two lines below to get rid of the numbers in the table of contents
%\let\Contentsline\contentsline
%\renewcommand\contentsline[3]{\Contentsline{#1}{#2}{}}

\begin{document}

%----------------------------------------------------------------------------------------
%	TITLE AND LIST OF QUESTIONS
%----------------------------------------------------------------------------------------

\begin{center}
\Huge{\bf \emph{A Template for FAQ's}} % Main title
\end{center}

\listofquestions % This prints the subtitle and a list of all of your questions

\newpage % Comment this if you would like your questions and answers to start immediately after table of questions

%----------------------------------------------------------------------------------------
%	QUESTIONS AND ANSWERS
%----------------------------------------------------------------------------------------

\question{Print via ssh}\label{new-question}

First we need to ssh into the machine on the network:
\begin{verbatim}
    igor@igor-HP:~$ ssh bogoslai@kiew.informatik.uni-freiburg.de
\end{verbatim}
This actually corresponds to the mask:
\begin{verbatim}
    <user>@<your_local_machine_name>:~$ ssh <username_on_remote_machine>@<name_of_remote_machine_on_network>.<domen>
\end{verbatim}
Then we would like to see which printers are on the network:
\begin{verbatim}
    [bogoslai@kiew ~]$ lpstat -p -d

    printer hp04 is idle.  enabled since Fri 26 Jul 2013 11:49:25 AM CEST
    printer hp14 is idle.  enabled since Fri 26 Jul 2013 01:57:44 PM CEST
        Ready to print.
    printer hp15 is idle.  enabled since Fri 26 Jul 2013 02:04:53 PM CEST
        Ready to print.
    printer hp17 is idle.  enabled since Fri 26 Jul 2013 01:52:15 PM CEST
        Ready to print.
    printer hpcolor is idle.  enabled since Fri 26 Jul 2013 01:38:33 PM CEST
        Ready to print.
    system default destination: hp15
\end{verbatim}
Now we are actualy able to choose the printer we want to use and print:
\begin{verbatim}
    lp -d hpcolor Example.pdf 
\end{verbatim}
The mask of this is:
\begin{verbatim}
    lp -d <name_of_printer> <filename> 
\end{verbatim}
%----------------------------------------------------------------------------------------

\end{document}